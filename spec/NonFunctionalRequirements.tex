\svnid{$Id: NonFunctionalRequirements.tex 1 2007-12-12 17:37:26Z t-scheller $}
\section{Non-functional requirements} \label{nf:Non-functional requirements}
\subsection{Technologies and development environment}
The following software is used for development:
\begin{itemize}
\item Java 5 SE
\item Eclipse 3.3.x
\item the customer's COBOL-85 grammar \footnote{bruessel.informatik.uni-stuttgart.de:/home/export/bruessel/projects/stupro06/grammars/cobol.jj}
\item Subversion 1.3.0 on the server for configuration management
\item ArgoUML 0.22 for use case diagrams in the \gl[specification]{specification}
\item BOUML 2.21.5 or compatible for UML diagrams

\end{itemize}
\par
The following technologies are used for development:
\begin{itemize}
\item LaTeX as document format
\item UTF-8 for encoding text
\item XML as the intermediate format for reports
\item Java and COBOL-85 example programs
\end{itemize}

\subsection{Requirements to the working environment}
\subsubsection{Software requirements}
%requirements to the hard- and software at the users computer
%JUnit not required but optional: JUnit for JUnit integration ~> no need to mention
The following programs are required to use the software:
\begin{itemize}
\item Java version 5 or later
\item Eclipse 3.3.x for all GUI functions
\item \gl[PDF]{PDF} viewer for documentation which supports at least PDF 1.4
%TODO: which version of JUnit? ; Iteration 3
\item JUnit for automatic \gl[test case]{test case} recognition
\item for COBOL support a COBOL-85 preprocessor to prepare code for \gl[instrumentation]{instrumentation} and a compiler to compile the instrumented code
\end{itemize}

\subsubsection{Hardware requirements} \label{nf:Hardware requirements}
%NOTE: customer mentioned no clear requirements: basically we may assert recent hardware
To support a wide installation base moderate hardware should be enough for using the software. Exact minimum requirements must be determined based on the final application.
\par
The minimum hardware required is:
\begin{itemize}
\item 512 MiB \footnote{1 MiB = $2^{20}$ Bytes} of RAM
\item a CPU as powerful as an AMD Athlon with 1 GHz clock rate
\item 100 MiB of free hard disk space for installation with Eclipse already installed, to store working data and for some \gl[session container]{session containers}
\end{itemize}

The following hardware is recommended:
\begin{itemize}
%min 512 MiB is suggested for eclipse
\item 1 GiB of RAM
%min 1 GHz is suggested for eclipse
\item a CPU as powerful as single core AMD Athlon with 2 GHz clock rate
%normally not mentioned, recent single drives achive 90 MB/s
%Johannes has ~250MB/s at home if necessary, only with 1.6GHz CPU however. 60MB/s via NFS and a fast dual core CPU could be made available
\item 60 MiB/s read and write sequential transfer rate measured at file system level
%would make ~1000 bytes per LOC when analyzing 5 Mio LOC
\item 10 GiB of free hard disk space to store working data and some coverage results
\end{itemize}

\subsection{Quantity requirements} \label{nf:Quantity requirements}
Quantities that have no defined maximum are only limited by the resources of the PC the software runs on.

\subsubsection{Program examples}
\paragraph{Small program} \label{nf:Small program}
\linkwithfootnote{http://sourceforge.net/projects/fred}{Fred v1.3.5 RC2} is a small program. All functions of the software that dont' work with it are completely useless.
\paragraph{Medium program} \label{nf:Medium program}
\linkwithfootnote{http://tomcat.apache.org/download-55.cgi}{Tomcat 5.5.20} is a medium sized program. All functions of the software must work with it.
\paragraph{Large program} \label{nf:Large program}
%is newest archived Version
\linkwithfootnote{http://archive.eclipse.org/eclipse/downloads/drops/R-3.1.2-200601181600/index.php}{Eclipse SDK 3.1.2} is a large program. Running functions on the large program is sufficient to show that they work for any \gl[project]{project} that must be supported.

\subsubsection{Timestamp}
Timestamps have a resolution of one minute or finer. The software must use a state of the art representation of timestamps to support a sufficiently wide time period.

\subsubsection{Text value}
Text values are of arbitrary length and may contain any Unicode characters.

\subsubsection{Test case}
The name and the comment of a \gl[test case]{test case} are text values. The start time is a time stamp.

\subsubsection{Test session}
An arbitrary number of \gl[test session]{test sessions} must be supported.
\par
The name and the comment of a \gl[test session]{test session} are text values. The start time is a time stamp.

\subsubsection{Code Base}
An arbitrary number of \gl[code base]{code bases} must be supported. The date and time of the first instrumentation is a time stamp.

\subsection{Performance requirements} \label{nf:Performance requirements}
All performance requirements must be met on the recommended hardware.

\subsubsection{Batch processing}
Building and instrumenting a large program (\ref{nf:Large program}) must be possible within 10 hours.

\subsubsection{Eclipse plug-in}
The following constraints must hold true for a medium program (\ref{nf:Medium program}) and should hold true for a large program (\ref{nf:Large program}).
\par
The Eclipse plug-in may not slow down Eclipse significantly.
\par
Additionally the following response times must be met. They don't apply to the first call of a function. During the first call initialization routines may add a significant delay.
\par
For simple GUI interaction 0.1 s response time is enough while 0.5 s is too slow. Selecting instrumentable items is such a simple GUI interaction.
\par
For interaction resulting in complicated rebuilds of the UI components the code highlighting 5 s response time is enough while 30 s is too slow. During this time the Eclipse UI may not respond. An example for such interaction is changing the source code annotations while they are displayed.
\par
For loading and saving files as well as processing tasks like building the correlation matrix and generating a report there are no performance requirements. These tasks may not block the Eclipse UI.

\subsection{Availability}
There are no special availability requirements.

\subsection{Security}
There are no special security requirements.

\subsection{Robustness and failure behavior} \label{nf:Robustness and failure behavior}
%how reacts the software if the puts in wrong inputs,
%e.g. a string in a date field
%what happens if Prof. Ludewig takes a seat on the keyboard
File types are detected using a syntax check for plain text files and at least 64 bit long magic numbers for binary files. That syntax check only has to detect wrong file types.
\par
The software may show arbitrary behavior when it runs on broken hardware.
%does not crash if input file is too big but shows errormessage; could be too difficult to catch any Memory Exception; TODO: Really required?

\subsection{Usability} \label{nf:Usability}
Any string displayed by the Eclipse plug-in is localizable. The default language is English.
\par
A \gl[developer]{developer} can install the software easily. The installation procedure may only require unzip and Java. The software must be installable within 5 minutes by a developer after he has found the installation instructions and downloaded the necessary files. %eases testing and spreading of the software
\par
All colors used for highlighting can be changed by the user.
\par
Internal procedures must be invisible to the user as long as he doesn't want to change their behavior. Where applicable sensible defaults must be preset.
\par
An English user's manual is required. Online help is not required.
\par
The user interface must be intuitive to the point that on line help is not needed. \linkwithfootnote{http://www.eclipse.org/articles/Article-UI-Guidelines/Index.html}{Eclipse User Interface Guidelines version 2.1} must be followed for the Eclipse plug-in. Additionally per default the user is asked in a confirm dialog before a file with valuable user data is deleted or overwritten. 
% Such dialogs have a checkbox to remember the answer permanently and not show the dialog again.
%"valuable user data" - exact enough?- example would be a session container

\subsection{Portability}
The software must run on all platforms Eclipse 3.2.x runs on.
\par
The delivered \gl[instrumentation]{instrumentation} procedure must be compatible with Java 2 version 1.4.x and Java 5 as well as preprocessed COBOL-85. Other languages must be implementable without changing the \gl[session container]{session container's} format.
\subsection{Maintainability}
The source code follows a style guide based on Sun's
\linkwithfootnote{http://java.sun.com/docs/codeconv/}{Code Conventions}. The
style guide is described in a separate document. 
\par
All technical documents made specifically for creating and verifying the software are released to the public.

\subsection{Extensibility} \label{nf:Extensibility}
The software is written to be highly extensible and documented on a level easy to understand for their target audience. The documentation of the Eclipse plug-in is written for the Eclipse user, the documentation of the batch interface is written for the shell user (see Actors \ref{fr:Actors}) and the documentation to extending the Software is written for \gl[maintenance engineer]{maintenance engineers}. Tutorials for maintenance engineers will show how to extend the software to other programming languages than COBOL and Java, how to add the collecting of metrics during the \gl[instrumentation]{instrumentation} and how to implement further coverage criteria. The tutorials may assume good knowledge of Java, Grammars, JavaCC and the languages that must be supported by the changed software.
\par
Reports can be customized using templates, which can be selected in Eclipse using a wizard. It must be easy to add new report formats like \gl[PDF]{PDF} and \gl[DocBook XML]{DocBook XML}, which can be transformed into many formats.
\par
To ease external analysis and report generation all data of a \gl[test session]{test session}, except for the boolean values sampled in \gl[conditional statement]{conditional statements}, must be available to easily implement an export to XML. The design team must decide if the \gl[session container]{session container} already has such a format, or if report generation already contains it as an intermediate step.
\par
Other coverage criteria should be easy to add for \gl[maintenance engineer]{maintenance engineers} as long as they
don't depend on execution order. No change to the \gl[instrumentation]{instrumentation}, test case management and highlighting may be necessary to implement another condition coverage criterion.
\par
%boolean and coverage analyzer, hot path visualisation and a correlation matrix
It must be easy to add further analysis of a session container. All evaluation results of boolean expressions and counters must be easily accessible.
\par
Based on the TestCaseNotifier class (see \ref{fr:Test sessions and test cases}), functions to define test cases must be added: a live mode with a graphical dialog which allows the user to start and stop test cases during the \gl[SUT]{SUT} execution and automatic test case recognition of JUnit test cases.
