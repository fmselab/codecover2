% \lstinputlisting{build-codecover.xml}

\section*{Anleitung zur Verwendung von CodeCover im SoPra}

%CodeCover ist ein Werkzeug zur Ermittlung von Überdeckungsdaten.

Diese Anleitung enhält eine Beschreibung zur Verwendung im Zusammenhang mit
\linkwithfootnote{http://ant.apache.org/}{Apache~ANT}.
Als Vorlagen für ANT-Build-Dateien existieren die Dateien
\code{build-codecover.xml} und \code{build-codecover2.xml}.
\code{build-codecover.xml} ist ein einfaches Beispiel das den Quellcode
instrumentiert, compiliert, ausführt und einen Report erzeugt.
Bei \code{build-codecover2.xml} wird der kompilierte Code zweimal ausgeführt
und das Ergebnis der beiden Ausführungen in einen Report geschrieben.

Zur Verwendung muß nur noch der Kopf der Dateien angepasst werden.

\begin{verbatim}
  <property name="codecoverDir" value="codecover" />
  <property name="sourceDir" value="src" />
  <property name="instrumentedSourceDir" value="instrumented" />
  <property name="mainClassName" value="Test" />
\end{verbatim}

Die Eigenschaft \code{codecoverDir} muß auf das CodeCover-Verzeichnis gesetzt
werden (also auf das Verzeichnis, das das Unterverzeichnis \code{lib} und darin
Dateien wie \code{codecover-core.jar} enthält).

Die Eigenschaft \code{sourceDir} muß auf das Verzeichnis gesetzt werden, das
die Java-Quellcode-Dateien enthält.

Die Eigenschaft \code{instrumentedSourceDir} muß auf das Verzeichnis gesetzt
werden, das später die instrumentierten Quellcode-Dateien und die kompilierten
Dateien enthält. Der Standardwert \code{instrumented} dürfte in den meisten
Fällen passen.

Die Eigenschaft \code{mainClassName} muß auf den Namen der Klasse gesetzt
werden, die ausgeführt werden soll.


Wenn also beispielweise CodeCover in \code{/home/foo/codecover} liegt,
die Quelldateien im Unterverzeichnis \code{src} im gleichen Verzeichnis wie
die Datei \code{build-codecover.xml} liegt und die Hauptklasse
\code{foo.Bar} heißt sollte der Kopf danach folgendermaßen aussehen:

\begin{verbatim}
  <property name="codecoverDir" value="/home/foo/codecover" />
  <property name="sourceDir" value="src" />
  <property name="instrumentedSourceDir" value="instrumented" />
  <property name="mainClassName" value="foo.Bar" />
\end{verbatim}

Danach kann das Build-Skript mit \code{ant -f build-codecover.xml} genutzt
werden.
