\svnid{$Id: Introduction.tex 1 2007-12-12 17:37:26Z t-scheller $}
\section{Introduction} \label{Introduction}

\subsection{Project overview} \label{in:Overview}
\gbt is a glass box testing tool to measure the coverage of a running program.
It will be as independent as possible of the programming language of the covered program.

%TODO: Bla Bla bla...
\subsection{About this document} \label{in:AboutThisDocument}

This document contains the necessary information to create a detailed design
obeying the requirements documented in the specification. It describes the
behaviour of the software and the artifacts which are created by it. The
architecture is also defined by this document, as well as the data structures
used to store the data pertaining to the operation of the software. Details
like the structure of classes and the design of methods have to be determined
in the detailed design document.


\subsection{Addressed audience} \label{in:Addressed audience}
This document is addressed to
\begin{itemize}
  \item the customer who ordered the software
  \item the project manager controlling the work
  \item the quality assurance division creating test cases for the software
  \item the developers implementing the design
  \item future developers maintaining and extending the software
  \item interested users of the software
  \item students of upcoming student projects
\end{itemize}

\subsection{Authors}
In the following table authors of this document are named.
{\small
\begin{longtable}{|p{35mm}|p{65mm}|l|} \hline
   {\normalsize \textbf{Author}} &
   {\normalsize \textbf{E-mail}} \\\hline \hline \endhead
   Robert Hanussek & \email{hanussrt@studi.informatik.uni-stuttgart.de} \\\hline
   Steffen Kieß & \email{kiesssn@studi.informatik.uni-stuttgart.de} \\\hline
   Tilmann Scheller & \email{schellrt@studi.informatik.uni-stuttgart.de} \\\hline
   Markus Wittlinger & \email{wittlims@studi.informatik.uni-stuttgart.de} \\\hline
\end{longtable}
}




\subsection{Notation}

\subsubsection{Identifiers}

\cls{FooBar} denotes a class named \code{FooBar}.

\clsab{FooBar} denotes an abstract class named \code{FooBar}.

\obj{FooBar} denotes an instantiated object of class \cls{FooBar}.

\itf{FooBar} denotes an interface class named \code{FooBar}.

\imp{FooBar} denotes an object of a class which implementes the interface named \code{FooBar}.

\mtd{fooBar(...)} denotes a method named \code{fooBar} which has parameters (which are not listed). Methods don't have a special color because they can be easily identified by the trailing brackets.

\mtd{fooBar()} denotes a method named \code{fooBar} which doesn't have any parameters.

\mtdst{fooBar(...)} denotes a static method (which has parameters).

\fld{fooBar} denotes a field named \code{fooBar}.

\pkg{foo.bar} denotes a package called \code{bar} which resides in a package called \code{foo}. Packages don't have a special color because they can be identified easily since they are the only items which contain a dot and are lower case.

%% \subsubsection{Namespace}

%% Every section can define a namespace. All relative paths in such a section are relative to this namespace except paths which point to standard Java classes which are listed below. If for example the namespace \pkg{\rootpkg.report} is defined then \pkg{html.}\cls{Report} points to class \cls{Report} which resides in the package \pkg{\rootpkg.report.html}. It is also possible to define namespaces which refer to classes, e.g. \pkg{\rootpkg.report.}\cls{Report}. This allows it to directly reference the methods in \cls{Report} (see~\ref{Classes:Report:Report} for an example).

%% Namespaces are defined directly under the header of a section and are valid for the section the header belongs to. A Namespace definition looks like this:
%% \namespace{\pkg{\rootpkg.report}}

%% Following standard Java classes are used in this document without specifing their fully qualified package paths. These classes are also unaffected by namespace definitions:
%% % we will just write String instead of java.lang.String
%% \begin{itemize}
%% \item \pkg{java.lang.}\cls{String}
%% \item \pkg{java.io.}\cls{OutputStream}
%% \item \pkg{java.util.}\itf{List}
%% \end{itemize}
