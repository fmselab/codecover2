\documentclass[a4paper,12pt,liststotoc,DIV12]{scrartcl}
\usepackage{longtable}
\usepackage[T1]{fontenc}
\usepackage{ae}
\usepackage[utf8]{inputenc}
\usepackage{fancyhdr}
\usepackage{graphicx}
\usepackage{amsmath}
\usepackage[
  %plainpages=false,
  %pdfpagelabels,
  %colorlinks=true,
  pdfborder={0 0 0},
  %urlcolor=blue,
  %linkcolor=blue,
  pdftitle={OST-WeST - Coding Guidelines},
  pdfsubject={OST-WeST - Coding Guidelines},
  pdfauthor={Stefan Franke, Robert Hanussek, Benjamin Keil, Steffen Kieß, Johannes Langauf, Christoph Marian Müller, Igor Podolskiy, Tilmann Scheller, Michael Starzmann, Markus Wittlinger},
  pdfkeywords={OST-WeST, Code, Coding, Guidelines, Styleguide},
  bookmarksopen=true,
  %pdfstartpage=3,
  %unicode,
]{hyperref}
\usepackage{svnkw}
\usepackage[USenglish]{isodate}
%\usepackage{titlesec}

\makeatletter

% some TeX voodoo to extract the date from SVN ID
% which can be processed by isodate 
\def\svn@dateonly#1 #2Z{#1}
\def\svndateonly#1{%
\ifx#1\empty1970-01-01\else
\expandafter\svn@dateonly#1\fi}

% Two-level figure & table numbering
\@addtoreset{table}{section}
\@addtoreset{figure}{section}
\renewcommand{\thefigure}{\thesection.\arabic{figure}}
\renewcommand{\thetable}{\thesection.\arabic{table}}
\makeatother

% alternate 
%\titleformat{\paragraph}{\bfseries\large\sffamily}{\theparagraph}{1em}{}
%\titleformat{\subparagraph}{\bfseries\sffamily}{\thesubparagraph}{1em}{}
%\titlespacing{\paragraph}{0pt}{\parskip}{-0.5\parskip}{}
%\titlespacing{\subparagraph}{0pt}{\parskip}{-0.5\parskip}{}

\svnid{$Id: coding-sg.tex 1 2007-12-12 17:37:26Z t-scheller $}

\parindent0mm
\parskip2mm
%\geometry{textwidth=160mm, textheight=230mm, inner=30mm}

%\xdefinecolor{TodoColor}{rgb}{1.0, 0.0, 0.0}

% depth of the headlines that are numbered
\setcounter{secnumdepth}{7}
\setcounter{tocdepth}{2}

\newcommand{\OSTWeST}{\textit{OSTWeST}}
\newcommand{\gbt}{\textit{gbt$^2$}}
\newcommand{\eclui}{\textsf}
\newcommand{\code}{\texttt}
\newcommand{\todo}[1]{\bgroup\color{TodoColor}\textsc{\textbf{TODO:} #1}\egroup}
\newcommand{\BIG}{\fontsize{48}{48}\selectfont}
\newcommand{\linkwithfootnote}[2]{\href{#1}{#2}\footnote{\url{#1}}}

% needed for the Function Point tables
\newcommand{\x}{\textbullet}

\begin{document}

% --- title page --- %
\pagestyle{empty}
\begin{titlepage}
 \vspace*{38mm}
 \begin{center}
 \fontsize{24}{24}\selectfont
 Coding Guidelines\\
 \vspace*{12mm}
 \fontsize{48}{48}\selectfont
 
 % gross hack to generate the a big fancy gbt-squared appearance
 \textit{gbt}$\fontsize{32}{32}\fontfamily{ptm}\selectfont ^\textit{2}$
 %
 \\
 \fontfamily{\familydefault}\fontsize{32}{38}\selectfont
 Glass Box Testing Tool\\
 \vspace*{12mm}
 \fontsize{16}{20}\selectfont
 Student Project A ``OST-WeST''\\
 University of Stuttgart

 \vspace{2cm}
 {\small 
   Version: 1.2 \\
   {Last changed on \printdate{\svndateonly{\svndate}} (SVN Revision \svnrev)}}
   \end{center}
   \vspace{3cm}
   \hspace{40mm}
   \normalsize
\end{titlepage}

% --- Header and version history --- %

\cleardoublepage
\fancyhf{}
\fancyhead[RE,LO]{\textit{\gbt\ - Coding Guidelines}}
\fancyhead[RO,LE]{\thepage}
\pagestyle{fancyplain}

% -- Version History -- %
%\svnid{$Id: VersionHistory.tex 23 2008-05-28 11:39:25Z ahija $}

\section*{Version History}

{\small
\begin{longtable}{|l|l|p{35mm}|p{71mm}|} \hline
   {\normalsize \textbf{Date}} &
   {\normalsize \textbf{Version}} &
   {\normalsize \textbf{Author}} & 
   {\normalsize \textbf{Modifications}} \\\hline \hline \endhead
    11.01.2007 &  0.1 &  Stefan Franke & 
      - Chapter files and master document file \\\hline
    16.01.2007 &  0.2 &  Stefan Franke & 
      - ui: Eclipse Plug-in and images within \\\hline
    17.01.2007 &  0.3 & Michael Starzmann \newline Christoph Müller & 
      - Headwords following the corresponding chapter in the analysis-notes \newline
      - nr: Keywords taken over by the analysis notes \newline
      - fr: The foreword for the functional requirements \\\hline
    18.01.2007 &  0.4 &  Stefan Franke & 
      - ui: Rewrite source code highlighting \newline
      - Reorder chapters \newline
      - Rename some sections \newline
      - ui: Added source code highlighting for COBOL \\\hline
    19.01.2007 &  0.5 &  Christoph Müller & 
      - Document structure changed \newline
      - fr: Use case pictures imported \newline
      - fr: Foreword, fr: actors, fr: general arrangements \\\hline
    20.01.2007 & 0.6 & Johannes Langauf \newline Christoph Müller & 
      - nf: Expand some keywords to complete sentences \newline
      - nf: Find new NFRs \newline
      - fr: Configuration \newline
      - fr: Use case description \newline
      - fr: Language support \\\hline
    21.01.2007 & 0.7 & Christoph Müller \newline Stefan Franke & 
      - fr: Use case description \newline
      - ui: Session view, Coverage view and Launching \\\hline
    23.01.2007 & 0.8 & Christoph Müller \newline Stefan Franke \newline Michael Starzmann& 
      - Correction after specification meeting \newline
      - fr: Use case description of measure coverage \newline
      - fr: General functional requirements \newline
      - fr: Reports \newline
      - ui: Configuration dialogs \\\hline
    24.01.2007 & 0.9 & Michael Starzmann \newline Christoph Müller & 
      - in: Introduction \newline
      - fr: Use case description \newline
      - fr: Coverage Criteria \\\hline
    25.01.2007 & 0.10 & Michael Starzmann \newline Stefan Franke & 
      - ui: Package and file selection to ... states \newline
      - fr: Coverage measurement improved \newline
      - in: Introduction \\\hline
    26.01.2007 & 0.11 & Christoph Müller \newline Stefan Franke & 
      - Correction after specification meeting \newline
      - fr: New use case instrument instrumentable items \newline
      - ui: Configuration sections \newline
      - ui: Source code highlighting \\\hline
    27.01.2007 & 0.12 & Christoph Müller \newline Stefan Franke \newline Michael Starzmann & 
      - fr: Use case description of administrate sessions \newline
      - ui: Instrumentation subsection \newline
      - ui: Solved todos \newline
      - ui: first draft for the Batch interface \newline
      - fr: Release, folders, files \\\hline
    28.01.2007 & 0.13 & Christoph Müller \newline Stefan Franke \newline Johannes Langauf & 
      - fr: Batch interface \newline
      - ui,fr: Correction after internal review \newline
      - nr: improve and fill out most non-functional requirements \\\hline
    29.01.2007 & 0.14 & Christoph Müller & 
      - fr: Solve todos, use case diagrams, folder structure \newline
      - Correction after QA meeting \newline
      - small spell check \\\hline
    30.01.2007 & 0.15 & Christoph Müller \newline Stefan Franke \newline Johannes Langauf & 
      - fr: New use cases: analyse coverage log, export session \newline
      - ui: New Context menu \newline
      - ui: Import, Export, Report \newline
      - ui: Small adaption at figures \newline
      - nf: Correction after specification review \newline
      - nf: Extensibility, performance requirements, program examples \\\hline
    31.01.2007 & 0.16 & Christoph Müller \newline & 
      - Correction after Igor's big bang \\\hline
    31.01.2007 & 1.0 & Igor Podolskiy & Declaring version 1.0, ready for review \\\hline
    08.02.2007 & 1.1-dev-1 & Stefan Franke \newline Michael Starzmann & 
      - Correction after specification review \\\hline
    09.02.2007 & 1.1-dev-2 & Christoph Müller & 
      - Correction after specification review \\\hline
    10.02.2007 & 1.1-dev-3 & Christoph Müller & 
      - Correction after specification review: Bugs 47, 90, 52, 57, 56, 58, 60, 62, 63, 64, 65, 39, 40, 41, 42, 44, 46, 50, 51, 52, 54, 34 \\\hline
    11.02.2007 & 1.1-dev-4 & Johannes Langauf \newline Christoph Müller \newline Stefan Franke & 
      - Correction after specification review: Bugs 48, 35, 32, 91, 16 \\\hline
    12.02.2007 & 1.1-dev-5 & Johannes Langauf \newline Christoph Müller  &
      - Correction after specification review: Bug 22 \newline
      - new batch commands \\\hline
    13.02.2007 & 1.1-dev-6 & Stefan Franke \newline Christoph Müller &
      - moved the glossary to specification document \newline
      - added links to glossary entries \newline
      - Bug 7: Work flow \newline
      - Bugs 6, 21, 28, 88, 89, 91\\\hline
    14.02.2007 & 1.1-dev-7 & Stefan Franke \newline Christoph Müller \newline Michael Starzmann &
      - Bug 45 \\\hline
    16.02.2007 & 1.1-dev-8 & Stefan Franke \newline Christoph Müller &
      - Bug 45 \\\hline
    11.05.2007 & 1.1-dev-9 & Stefan Franke \newline Christoph Müller &
      - Bugs 98, 99, 100 \\\hline
    15.06.2007 & 1.1-dev-10 & Christoph Müller & 
      - fr: \ref{fr:JUnit integration} JUnit integration \\\hline
    17.06.2007 & 1.1-dev-11 & Stefan Franke & 
      - ui: \ref{ui:Boolean Analyzer} Boolean Analyzer \\\hline
    18.06.2007 & 1.1-dev-12 & Christoph Müller & 
      - fr: \ref{fr: Coverage measurement} coverage log file name \newline
      - fr: \ref{fr:Batch:Instrument} instrument supports a charset  \newline
      - fr: \ref{fr:Batch:Analyze} analyze supports a charset \newline
      - fr: \ref{fr:Batch:Instrumenter-info} Instrumenter-info \\\hline
    19.06.2007 & 1.1-dev-13 & Christoph Müller & 
      - fr: \ref{fr:Batch:Instrument} instrument has \verb$--$copy-uninstrumented \\\hline
    29.06.2007 & 1.1-dev-14 & Christoph Müller & 
      - fr: \ref{fr:Batch:Instrument} instrument has include, exclude \\\hline
    19.09.2007 & 1.1-dev-15 & Johannes Langauf & 
      - ui: \ref{ui:Hot-Path} Hot-Path: make outstanding decisions, update for configureable colors \newline
      - fr: remove PDF-Report support \\\hline
     31.10.2007 & 1.1-dev-16  & Tilmann Scheller  & general update of specification
       \\\hline
    28.05.2008 & 1.1-dev-17 & Christoph Müller & 
      - fr: \code{return} and \code{break} are basic statements too \\\hline
%     &   &   & 
%       \\\hline
\end{longtable}
}

%%% Local Variables: 
%%% mode: latex
%%% TeX-PDF-mode: t
%%% TeX-master: "Specification.ltx"
%%% End: 


% \cleardoublepage

% --- Table of contents --- %

\parskip1mm
\tableofcontents
\parskip2mm

\pagestyle{fancyplain}
\renewcommand{\baselinestretch}{1.25}\normalsize
\section{Introduction}
\label{sec:introduction}

This document specifies the coding guidelines for the development of \gbt. It
is intended primarily for Java source code but also for other files like XML
descriptors that depend on the code and are distributed with the source code.

\subsection{Conventions}
\label{sec:conventions}

The emphasized key words \emph{must}, \emph{must not}, \emph{required},
\emph{shall}, \emph{shall not}, \emph{should}, \emph{should not}, \emph{may},
\emph{recommended} and \emph{optional} are to be interpreted as described in
\linkwithfootnote{http://rfc.net/rfc2119.html}{RFC 2119}.


\subsection{Contact Person}
\label{sec:contact-person}

The contact person for this document is Igor Podolskiy
(\texttt{podolsir@studi.informatik.uni-stuttgart.de}).

\section{General Guidelines}
\label{sec:general-guidelines}


\subsection{Sun Code Conventions}
\label{sec:general:sun}
Any Java source code \emph{must} follow the general
\linkwithfootnote{http://java.sun.com/docs/codeconv/}{Sun Code Conventions for
  the Java Programming Language}.

\subsection{Language}
\label{sec:javadoc:language}
All comments (Javadoc as well as non-Javadoc) \emph{must} be written in English.

All identifiers that are based on natural language \emph{must} use
English as their base language.

\subsection{Spaces, No Tabs}
\label{sec:general:spaces-no-tabs}
All source code and code-like files (e.g. XML plugin descriptors) \emph{must}
use spaces instead of tabs for indentation to ensure portability between
platforms and editors.

\subsection{Subversion \texttt{\$Id\$} keyword}
\label{sec:general:subversion-id}
All text files of a format supporting comments \emph{must} contain a comment
with Subversion \$Id\$ keyword at the beginning of the file. The
\texttt{svn:keyword} property of the file \emph{must} be set to \texttt{Id} in
Subversion.

\subsection{No Spaces in Filenames}
File and directory names \emph{must not} contain any whitespace (spaces, tabs,
\verb!\r!, \verb!\n! and alike).

\subsection{Naming Rules}
\label{sec:general:naming-rules}

\subsubsection{General Naming Rules}
\label{sec:general-naming-rules}
Java identifiers \emph{must not} contain any characters other than basic Latin
letters (A-Z in upper or lower case) digits (0-9) and underscores
(\texttt{\_}).

\subsubsection{Packages}
\label{sec:naming:packages}
Java package names \emph{must not} contain any character other than lower
case characters and digits.

\subsection{Javadoc}
\label{sec:general:javadoc}

\subsubsection{Classes}
\label{sec:general:javadoc:classes}
All Java classes \emph{must} have a Javadoc comment.

\texttt{@author} and \texttt{@version} tags are \emph{required}. The
\texttt{@author} tag \emph{must} contain the full name(s) of the main author(s)
of the class. The \texttt{@version} tag \emph{must} contain a logical version
and \emph{should} contain Subversion \texttt{\$Id\$} information enclosed in
parentheses, for example:

\begin{verbatim}
   /**
    * [...]
    * @author Igor Podolskiy
    * @version 1.0 ($Id: coding-sg.tex 1 2007-12-12 17:37:26Z t-scheller $) 
    */
\end{verbatim}

\subsubsection{Members}
All public and protected members \emph{must} have a Javadoc
comment.

Package visible and private members \emph{should} have a Javadoc comment.

The \texttt{@param}, \texttt{@returns}, \texttt{@throws} and \texttt{@since}
tags are \emph{required} where applicable.

The \texttt{@author} tag \emph{should} be specififed if the author of the
specific method is different from the author of the class.

\subsubsection{Packages}
\label{sec:general:javadoc:packages}

All packages \emph{must} have a Javadoc comment in the
\texttt{package-info.java} file.

\subsubsection{Unit Tests}
\label{sec:unit-tests}

Unit test classes \emph{must} be named according to the widely used
pattern

\begin{verbatim}
    <ClassOrTopicBeingTested>Test
\end{verbatim}

If a unit test tests a single particular class, its class \emph{must}
be placed in the same Java package as the respective class under test.

If a unit test tests a specific topic or feature, it \emph{must} be named
accordingly and reside in the main component package.

For example, a unit test for the

\begin{quote}
  \texttt{org.gbt2.instrumentation.java.JavaInstrumenter}
\end{quote}

class would
have the fully qualified name

\begin{quote}
  \texttt{org.gbt2.instrumentation.java.JavaInstrumenterTest}
\end{quote}

A unit
test in the same component which is testing the COBOL statement
instrumentation would have the fully qualified name
\begin{quote}
  \texttt{org.gbt2.instrumentation.COBOLStatementTest}
\end{quote}

\subsection{File System Layout}
\label{sec:fs-layout}

Every component (as specified by the Design document) has its
directory in the source code repository named after the component. The
generic layout of a component directory is as follows:

\begin{tabular}{ll}
  \verb!instrumentation/! & name of the component\\
  \verb!   src/! & source folder with the main Java source files\\
  \verb!   junit/! & source folder with the unit test source files (see also \ref{sec:unit-tests})\\
  \verb!   bin/! & contains Java \texttt{.class} files (not CM controlled)\\
  \verb!   .project! & Eclipse project file\\
  \verb!   build.xml! & Ant build file (see \ref{sec:ant-build-files} for details)\\  
\end{tabular}

The \texttt{src} and \texttt{junit} directories \emph{must} directly contain
the package hierarchy.

\subsection{Ant Build Files}
\label{sec:ant-build-files}

The Ant build files \texttt{must} contain at least the follwing targets:

\begin{tabular}{ll}
  \texttt{compile} & compile the particular component\\
  \texttt{jar} & build the component \texttt{jar} file\\
  \texttt{test} & compile (if needed) and run unit tests\\
  \texttt{clean} & delete automatically generated (i.e. compiled) files\\
\end{tabular}


\section{Guidelines for Specific Components}
\label{sec:specific-components}

\subsection{Eclipse Plugins}
\label{sec:specific:eclipse-plugins}

\subsubsection{Interface Naming}
\label{sec:eclipse:interface-naming}

Interfaces that are part of Eclipse plugins \emph{must} have their name
starting with \texttt{I}, for example \texttt{IMyInterface} instead of
\texttt{MyInterface}, as this is the standard Eclipse coding practice.

\subsection{External Components Conformance Exception}
\label{sec:specific:external-components}

External components of which substantial parts (e.g. packages or multiple
classes) are used as a basis for further development within the scope of the
\gbt\ project \emph{may} be non-conformant to the rules specified in this
document. However, it is \emph{recommended} to modify the code to adhere to
these rules.

Please note that if only a small part code is adopted (e.g. a single method),
this exceptional regulation explicitly \emph{does not} apply. In such cases,
the code \emph{must} be modified to conform to these guidelines.

% --- List of figures --- %

%\newpage
%\listoffigures
%\addcontentsline{toc}{section}{List of figures}

\end{document}

%%% Local Variables:
%%% TeX-PDF-mode: t
%%% End: